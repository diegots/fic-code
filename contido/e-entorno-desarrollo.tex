%%%%                       %%%%
%%%% ENTORNO DE DESARROLLO %%%%
%%%%                       %%%%

\chapter{Entorno de desarrollo}
\label{chap:entorno-desarrollo}

En este capítulo se explica como es el entorno utilizado para llevar a cabo el proyecto. En particular sobre que sistema operativo y lenguajes de programación se han utilizado.

\lettrine{E}{l} entorno de desarrollo utilizado consistió en una distribución Ubuntu GNU/Linux con versiones actualizadas de las aplicaciones utilizadas en el proyecto. Para el desarrollo en lenguaje Python se optó por el IDE Eclipse con el plugin Pydev, específico para desarrollo con lenguaje Python. El codigo C y los scripts Bash fueron realizados con el editor Atom.

- Portátil Compaq con un i5 de segunda generación y 12GB de ram
- Ubuntu Gnu/Linux
- Eclipse con el plugin Pydev para desarrollo en Python
- Atom
- TeXstudio
- Zotero para las referencias con el plugin Better BibTeX (BBT) para gestionar las claves
- Dropbox Paper y Google Docs para la organización, borradores, documentación
- yEd, la herramientas para creación de disgramas de yWorks