%%%%                 %%%%
%%%% INTRODUCCIÓN    %%%%
%%%%                 %%%%

\chapter{Introducción}
\label{chap:introduccion}

Aunque las empresas realizan un esfuerzo por sumarse a la digitalización, sigue siendo habitual que no existan canales de comunicación estandarizados para el intercambio de información con clientes, proveedores o trabajadores. El modelo tradicional se basa en la utilización de documentos en papel como base para dar evidencia de las operaciones. Suponiendo, por ejemplo, un escenario donde una empresa realiza una compra a un proveedor, para completar la entrega del producto, se utiliza un albarán que el transportista presenta en la empresa y se lleva firmado tras el depósito. El cliente se queda con una copia del documento y la operación finaliza o pasa a una siguiente fase, como puede ser el pago de la mercancía. ¿Pero, qué sucede luego con estos documentos? Deben ser archivados como prueba del intercambio y la información tiene que estar disponible en los sistemas de la empresa para llevar el control de gastos, volumen del stock, etc. Esta adquisición de información, al ser un documento en papel, implica un tratamiento manual. En otras situaciones, el mismo documento puede ser un PDF generado de forma digital pero que, nuevamente, es tratado de forma manual como única vía.

El personal de administración suele ser encargado de registrar toda la información pertinente y ello implica varios inconvenientes para la empresa. En primer lugar, debido a los posibles errores tipográficos en que se puede incurrir durante el proceso. No ayuda lo monótono de la tarea. Además, al tratarse de un proceso lento, la información no está inmediatamente disponible, lo que tiene un claro impacto durante la toma de decisiones con conocimiento incompleto de las situaciones. El coste de este modelo crece con el número de documentos que han de ser procesados. Esto se deriva del hecho de que existe una cantidad máxima de trabajo que una persona puede hacer en una jornada y la única manera de escalar es aumentando el personal dedicado a estas tareas.

El problema no se limita al intercambio de albaranes, facturas o formularios de los departamentos de Recursos Humanos. Existen todo tipo procesos que implican a un documento y donde además este documento tienen un modelo fijo. Estos modelos tienen características comunes como celdas, tablas, casillas o pares de tipo clave-valor. La única variación entre dos ejemplares distintos está en los datos pero no en su estructura. Esto abre la oportunidad de crear una solución capaz de procesar ejemplares de estos modelos de forma automática, siempre y cuando la solución tenga conocimiento de la distribución y tipo de información representada. 

Mientras la integración de los sistemas no evite Esto no implica cambiar el proceso actual sino conseguir explotar el soporte existente y así minimizar los costes para la empresa. 


\section{Motivación}

Son varias las motivaciones para proponer este proyecto. El formato PDF es un rotundo éxito comercial y consecuencia de ello es la ubicuidad que presenta en todos los ámbitos de la sociedad. Pese a ello, es poco conocida la manera en que la información está almacenada en el formato o por qué no es posible extraer información estructurada directamente de él. En este sentido el proyecto permitirá ahondar y comprender mejor esta estructura en la que confiamos para que en un futuro sus contenidos permanezca accesibles.

Por otra parte, esta es una oportunidad para profundizar más en algunos de los conocimientos adquiridos en el transcurso de la carrera como son los Procesadores de Lenguajes o las técnicas de Visión por Computador.

Por último, se espera lograr adquirir conocimientos necesarios para lograr una solución que pueda llegar a madurar para convertirse en una propuesta comercial interesante y que evolucione la manera en que las empresas adquieren información de sus procesos rutinarios. No hay que olvidar que existen otros competidores que llevan tiempo trabajando en este ámbito.

\section{Objetivos} 

Presentado el problema, los objetivos concretos son varios. En este proyecto se propone afrontar el problema en tres fases separadas. De forma general, se comenzaría realizando la extracción del texto por medio de Reconocimiento Óptico de Caracteres o de forma directa si los PDF tienen un origen digital y lo permiten. El siguiente paso implicará seleccionar la información relevante a partir de plantillas que delimitan las regiones de interés y tipo de información. Por último estandarizar la salida empleando para ello un formato estructurado como el JSON. Para lograrlo, las Tecnologías de Procesamiento de Lenguajes ayudarán a modelar la información, filtrar los detalles innecesarios y generar la salida deseada.

\begin{itemize}
    \item A partir de un conjunto de documentos de trabajo se identificarán las regiones y tipo de información que contienen.
    \item Del conocimiento de los documento se generarán plantillas que especifiquen las partes relevantes.
    \item Aunque se ha comentado de forma general como será el producto final, será necesario resolver ciertas incógnitas, como por ejemplo, cómo obtener información de coordenadas para emparejar el contenido con las plantillas.
    \item Crear una herramienta backend capaz de realizar un procesamiento por lotes automático
    \item Tratar los documentos disponibles para demostrar la herramienta. Se incluirán tipos de documentos basados en texto y en imagen.
    \item Posteriormente se harán las adaptaciones necesarias para facilitar la integración con frontend web.
\end{itemize}

\section{Estructura de la memoria}

En esta sección se presenta la estructura del resto del documento.

\begin{enumerate}
    \item \textbf{Introducción}. Es el capítulo actual. En el se presenta el problema que se va a tratar y las motivaciones para realizar en este Trabajo Fin de Grado.
    \item \textbf{Estado del arte}: el capítulo \ref{chap:estado-arte} está dedicado a mostrar algunas de las soluciones comerciales existentes en el mismo ámbito de aplicación que este trabajo y entender cual es su propuesta de valor.
    \item \textbf{Bases teóricas}. En el capítulo \ref{chap:bases-teoricas} se explica en que pilares teóricos se apoya la solución para lograr sus objetivos.
    \item \textbf{Metodología}. El capítulo \ref{chap:metodologia} se explica cual es la metodología escogida para completar el proyecto en tiempo y forma.
    \item \textbf{Fundamentos tecnológicos}. El capítulo \ref{chap:fundamentos-tecnologicos} de fundamentos tecnológicos presenta cuales son las aplicaciones y librerías utilizadas en el desarrollo.
    \item \textbf{Planificación}. La planificación explica tanto la propuesta inicial de planificación como los resultados finales. Una parte importante de este capítulo se dedica a relacionar la metodología con las fases individuales del proyecto. En este capítulo también comentan los costes económicos asociados al proyecto.
    \item \textbf{Análisis}. El capítulo \ref{chap:analisis} dedicado al análisis presenta casos de uso, requisitos y se presentan también los documentos tratados en el proyecto.
    \item \textbf{Implementación}. En el capítulo dedicado a la implementación se exponen los detalles técnicos de la construcción del software.
    \item \textbf{Conclusiones}. El último capítulo presenta los objetivos conseguidos y las lecciones apredidas durante la realización del trabajo.
\end{enumerate}
