%%%%                 %%%%
%%%% INTRODUCCIÓN    %%%%
%%%%                 %%%%

\chapter{Introducción}
\label{chap:introduccion}

La gestión de los documentos en las empresas es una tarea que consume tiempo, es propensa a errores, los datos no están inmediatamente disponibles. El volumen de documentos crece así como lo hace la organización. Los documentos se pueden agrupar dependiendo de su naturaleza, facturas, albaranes, formularios, órdenes de compra. Una vez definidas las agrupaciones se trata de crear un modelo que represente dicha familia de documentos y puede ser utilizado para capturar la información

\begin{itemize}
    \item Explicar qué son las familias de documentos. Problema principal que se quiere tratar en el proyecto.
\end{itemize}

\section{Objetivos del proyecto}

El proyecto persigue varios objetivos. A partir de la colección de documentos disponible y atendiendo a sus características, se busca encontrar un método para extraer información desde documentos PDF y que dicha información pueda adaptarse a un formato estructurado de forma automática. Como consecuencia de lo anterior se diseñará y construirá una aplicación para servidor que permita realizar todos los pasos automatizables del proceso. Se identificarán en los documentos posibles tipos de regiones tratables y se elaborarán plantillas con la información necesaria para ser utilizadas en el sistema. Para probar el software, se resolverán varios modelos del documentos, algunos de PDF basados en imagen y otros basados en texto. Posteriormente se realizarán a la aplicación las adaptaciones necesarias para facilitar la integración con una aplicación web.

\section{Estructura de la memoria}
