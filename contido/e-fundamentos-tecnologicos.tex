%%%%                          %%%%
%%%% FUNDAMENTOS TECNOLÓGICOS %%%%
%%%%                          %%%%

\chapter{Fundamentos tecnológicos}
\label{chap:fundamentos-tecnologicos}

\lettrine{E}n este capítulo se explica como es el entorno utilizado para llevar a cabo el proyecto. En particular se comentan las herramientas específicas y las tecnologías utilizadas.

\section{Herramientas para la realización del desarrollo}

La máquina de trabajo consistió en un portátil HP con un procesador Intel i5 de segunda generación, 12 GB de RAM y sistema operativo Ubuntu. Para el desarrollo en Python se utilizó Eclipse con el plugin Pydev
\footnote{https://www.pydev.org}. Los editores Atom y vim sirvieron para los scripts y el código en C de los escáneres y parsers. La memoria en \LaTeX está elaborada con TeXstudio. Se utilizó Zotero para las referencias y el plugin Better BibTeX (BBT) 
\footnote{https://retorque.re/zotero-better-bibtex} para gestionar las claves de la bibliografía. Los diagramas están hechos con yEd, la herramienta para creación de diagramas de yWorks 
\footnote{https://www.yworks.com/products/yed}. El control de versiones se llevó a cabo con git y Github. Dropbox Paper tiene la facilidad de representar notación Markdown con un diseño agradable así que fue utilizado para la organización, borradores, notas, etc.

\section{Tecnologías}

En esta sección se introducen las tecnologías utilizadas en el proyecto.

\subsection{OpenCV}

\textbf{OpenCV} \cite{opencvTeam_oficialSite_main} es una librería de fuente abierta para visión por computador y aprendizaje máquina. Desde su nacimiento en el año 1999 acumula más de 2500 algoritmos y muchas más funciones que hacen uso de ellos. El objetivo del proyecto es proveer infraestructura para facilitar la construcción de aplicaciones complejas de forma rápida. Es ampliamente utilizado en todo el mundo gracias a las facilidades de su licencia BSD. 

\begin{wrapfigure}{R}{0.3\textwidth}
    \centering
    \includegraphics[width=0.25\textwidth]{imaxes/e-fundamentos-tecnologicos/logo-opencv.png}
    %\caption{\label{fig:frog1}This is a figure caption.}
\end{wrapfigure}

Está escrita en C y C++ pero dispone además de interfaces de compatibilidad para varios lenguajes de programación como Java, Python o MATLAB. Se puede instalar en los principales sistemas operativos para ordenadores y también en Android. Los ámbitos de aplicación son múltiples, así como el número de empresas grandes y pequeñas que la explotan. Se pueden encontrar usos de OpenCV en sistemas de seguridad, análisis de calidad en fábricas, imagen médica, robótica, por citar algunos. En lo que respecta a este proyecto, se utiliza la versión optimizada del algoritmo de Hough que trae implementada.

\subsection{Flex}

\subsection{Bison}

\subsection{Docker}

\textbf{Docker} \cite{dockerincEmpoweringAppDevelopment} es una plataforma basada en estándares abiertos para facilitar el desarrollo, liberación y ejecución de aplicaciones.
Los contenedores se ejecutan aisladamente. Es un entorno virtualizado

- La premisa principal de Docker es aislar la aplicación de la infraestructura donde debe ejecutarse
- Docker propone metodologías para acelerar los procesos de publicación de software
- Es una tecnología de virtualización más ligera y rápida que la basada en hipervisores \footnote{Un hipervisor, \emph{hypervisor} en inglés, es un software utilizado para crear un ejecutar máquinas virtuales. Utilizando un hipervisor es posible para un ordenador que actúe de anfitrión, compartir recursos hardware con múltiples sistemas operativos invitados.}




\subsection{Ansible}

\subsection{Librería de generación de JSON en C}