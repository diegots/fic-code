%%%%                          %%%%
%%%% FUNDAMENTOS TECNOLÓGICOS %%%%
%%%%                          %%%%

\chapter{Fundamentos tecnológicos}
\label{chap:fundamentos-tecnologicos}

\lettrine{E}n este capítulo se explica como es el entorno utilizado para llevar a cabo el proyecto. En particular se comentan las herramientas específicas y las tecnologías utilizadas.

\section{Herramientas para la realización del desarrollo}

La máquina de trabajo consistió en un portátil HP con un procesador Intel i5 de segunda generación, 12 GB de RAM y sistema operativo Ubuntu. Para el desarrollo en Python se utilizó Eclipse con el plugin Pydev
\footnote{https://www.pydev.org}. Los editores Atom y vim sirvieron para los scripts y el código en C de los escáneres y parsers. La memoria en \LaTeX está elaborada con TeXstudio. Se utilizó Zotero para las referencias y el plugin Better BibTeX (BBT) 
\footnote{https://retorque.re/zotero-better-bibtex} para gestionar las claves de la bibliografía. Los diagramas están hechos con yEd, la herramienta para creación de diagramas de yWorks 
\footnote{https://www.yworks.com/products/yed}. El control de versiones se llevó a cabo con git y Github. Dropbox Paper tiene la facilidad de representar notación Markdown con un diseño agradable así que fue utilizado para la organización, borradores, notas, etc.

\section{Tecnologías}

\subsection{OpenCV}

\subsection{Flex}

\subsection{Bison}

\subsection{Docker}

\subsection{Ansible}

\subsection{Librería de generación de JSON en C}