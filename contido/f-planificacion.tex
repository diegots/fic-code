%%%%               %%%%
%%%% PLANIFICACIÓN %%%%
%%%%              %%%%

\chapter{Planificación}
\label{chap:planificación}

\lettrine{L}{a} creación de este TFG comparte las características generales de cualquier proyecto, ya que tiene un objetivo concreto y definido, no es una actividad rutinaria, tiene unas fechas de comienzo y de fin. Un proyecto necesita recursos en forma de tiempo, mano de obra, herramientas y conocimiento. En este capítulo se aborda la planificación elaborada.


\section{Recursos necesarios}

Para acometer el proyecto son necesarios varios tipos de recursos:

\begin{itemize}
    \item \textbf{Humanos}
        \begin{itemize}
            \item Un Product Owner, rol asignado al director del proyecto.
            \item Un analista.
            \item Un diseñador.
            \item Un equipo de desarrollo, formado íntegramente por el alumno.
        \end{itemize}
    \item \textbf{Software}: todos los recursos software utilizados son aplicaciones sin coste económico, mucho de ellos, además, software libre.
    \item \textbf{Hardware}: el portátil, pantalla y otros accesorios utilizados.
\end{itemize}

El rol de Product Owner lo realiza el director del proyecto, como se explicó en la Sección \ref{sec:uso-scrum-en-el-proyecto}. Las funciones de análisis, diseño y desarrollo fueron asumidas por el alumno. Cabe destacar que la colaboración para la creación de una interfaz gráfica a este proyecto fue realizada por otro alumno de la Facultad de Informática, no obstante debe quedar claro que no intervino en ningún momento en la elaboración de la parte del trabajo presentada en esta memoria.

\subsection{Planificación económica}

De los recursos utilizados para el proyecto se deducen directamente los costes económicos incurridos. En la Tabla \ref{tab:costes-proyecto} se presentan todos los gastos. Para el cálculo de los costes salariales se consultó el \emph{Convenio colectivo del sector de empresas de ingeniería y oficinas de estudios técnicos} en su página del BOE \footnote{\url{https://www.boe.es/diario_boe/txt.php?id=BOE-A-2019-14977}}.

\begin{table}[ht]
    \centering
    \begin{tabular}{l l l l}
        Recurso & Coste (€/h) & Tiempo (h) & Coste total (€) \\
        \hline
        \hline
        Jefe de proyecto & 42 & 18 & 756 \\
        Analista & 35 & 65 & 2275 \\
        Diseñador & 35 & 50 & 1750 \\
        Programador & 30 & 445 & 13350 \\
        Equipamiento informático & -- & -- & 900 \\
        Microsoft Project (licencia universitaria) & -- & - & 0 \\
		Otro software & -- & - & 0 \\    
        \hline
        \hline
        TOTAL & & - & 19031 \\        
    \end{tabular}
	\caption{Costes del proyecto}    
	\label{tab:costes-proyecto}
\end{table}

\section{Planificación temporal inicial}

En la Imagen \ref{fig:gantt-inicial} se presenta el diagrama de Gantt inicial de la planificación con el desglose de tareas. Una de las dificultades para la planificación consistió en no saber en qué momento del proyecto sería necesario abordar la integración con la interfaz web. Inicialmente se asumió que ocurriría durante el trabajo de desarrollo y antes de comenzar la escritura de la memoria. En este plan inicial se contemplaban 520 horas de trabajo.

\begin{figure}[hp!]
    \centering
    \includegraphics[angle=90,width=1.0\textwidth]{imaxes/f-planificacion/gantt-inicial.png}
    \caption{Diagrama de Gantt y tareas iniciales del proyecto}
    \label{fig:gantt-inicial}
\end{figure}

\section{Sprints}

Se expone a ahora la distribución del trabajo a lo largo de los Sprints. Por motivos de compatibilidad con el trabajo y responsabilidades personales, se escogió una fórmula de 20 horas de trabajo semanal y Sprints semanales. Al realizarse a media jornada, los Sprints tuvieron dos semanas reales de duración, completando de esta manera una jornada completa en cada uno.

\subsection{Sprint 1}

Este primer Sprint consistió por completo en un Spike con el objetivo de recabar toda la información necesaria para plantear el proyecto a nivel técnico. Algunos de los aspectos sobre los que se adquirió conocimiento fueron:

\begin{itemize}
    \item Alternativas comerciales en la misma temática.
    \item Herramientas y librerías disponibles en Linux para manipular PDF.
    \item Funcionamiento del formato PDF a nivel interno.
    \item Formatos de salida de los motores de OCR, en especial del seleccionado, hOCR.
    \item Posibles \emph{engines} de OCR candidatos para el proyecto
\end{itemize}

\subsection{Sprint 2}

Con el conocimiento adquirido se procedió a la construcción inicial del sistema para el procesamiento por lotes de los trabajos. Además de diseñar la estructura de directorios utilizada por el proyecto, se crearon el Makefile general, utilizado para las compilaciones y los \emph{scripts} para implementar el motor.

\subsection{Sprint 3}

En el Sprint 3 se crearon las implementaciones iniciales del generador de lenguaje intermedio y se preparó el primer \emph{parser}, que se probó con los documentos de Adobe. Para la construcción de los analizadores léxico-sintácticos se deseaba un enfoque modular, que no es la fórmula directa con este tipo de herramientas. Construir la herramienta siguiendo esta aproximación llevó más tiempo que el habitual en el caso de escoger un planteamiento más sencillo.

\subsection{Sprint 4}

En este Sprint se añadió soporte para la generación de código intermedio y \emph{parsing} del modelo de Adobe y parte del de OVH. También tuvieron lugar las primeras reuniones de trabajo para explicar el proyecto y decidir las adaptaciones necesarias de cara al prototipo.

\subsection{Sprint 5}

Se completó el modelo de documento del proveedor OVH. Además se abordaron de forma paralela las tareas relacionadas con el prototipo web. Hubo varias reuniones para facilitar la implementación de aspectos que no habían sido considerados, como  la necesidad de generar imágenes de las páginas individuales de cada documento, cambio del formato de la marca de tiempo para mejorar la compatibilidad entre sistemas de ficheros y la dockerización de la aplicación.

\subsection{Sprint 6}

El Sprint 6 se dedicó a realizar el modelo de Prodware y a finalizar la integración con la aplicación web.

\subsection{Sprints 7 y 8}

En estos dos Sprints se abordó la escritura inicial de la memoria, con la previsión de dejar tiempo al final del proyecto para completarla.

\subsection{Sprint 9}

Implementación del modelo de AC, basado en imagen.

\subsection{Sprint 10}

Implementación del modelo de AC, basado en imagen.

\subsection{Sprints 11 al 15}

Los últimos Sprints se dedicaron a completar la escritura de la memoria.

\section{Planificación temporal final}

La planificación final tuvo un desvío claro en la elaboración de la memoria, que finalmente llevó cuatro semanas más de las previstas (2 Sprints). Así mismo, para ir incorporando los sucesivos documentos fue necesario cierto retrabajo en la parte del generador de código intermedio, al tener que ampliar sus funcionalidades, manteniendo los resultados conseguidos hasta el momento. Las reuniones de integración sucedieron, aproximadamente, dentro del periodo esperado, antes de dar comienzo a la primera parte de la escritura de la memoria. El resultado final de la planificación se puede consultar en la Imagen \ref{fig:gantt-final}.

\begin{figure}[hp!]
	\centering
	\includegraphics[angle=0,width=1.0\textwidth]{imaxes/f-planificacion/gantt-final}
	\caption{Diagrama de Gantt del final del proyecto}
	\label{fig:gantt-final}
\end{figure}

Por último, en la imagen \ref{fig:tablero-github} se muestra parte del tablero de tareas creado en Github y que ha sido utilizado de referente para llevar el control de las tareas realizadas y pendientes durante la realización del proyecto.

\begin{figure}[hp!]
	\centering
	\includegraphics[angle=90,width=0.70\textwidth]{imaxes/f-planificacion/tablero-planificacion}
	\caption{Tablero en Github con las tareas del proyecto}
	\label{fig:tablero-github}
\end{figure}