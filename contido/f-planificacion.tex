%%%%               %%%%
%%%% PLANIFICACIÓN %%%%
%%%%              %%%%

\chapter{Planificación}
\label{chap:planificación}

\section{Recursos necesarios}

Para acometer el proyecto son necesarios varios tipos de recursos:

\begin{itemize}
    \item Recursos humanos
    \item Recursos software
    \item Recursos hardware
\end{itemize}

\section{Planificación económica}

\section{Planificación temporal inicial}

\section{Planificación económica}

\section{Sprints}

\subsection{Sprint 1: Spike, búsqueda de información, viabilidad}
\subsection{Sprint 2: estructura de directorios, creación del engine}
\subsection{Sprint 3: primer modelo}
\subsection{Sprint 4: segundo modelo}
\subsection{Sprint 5: modelo n}
\subsection{Sprint 6: adaptaciones para el MVP}
\subsection{Sprint 7: escritura de la memoria}
\subsection{Sprint 8: modelo n+1 para un caso OCR}


% TODO explicación del hardware utilizado
% TODO costes hardware
% TODO costes software

%% \section{Resultados de la planificación (temporal, económica)}
%% 
%% \section{Planificación económica}
%% 
%% La metodología utilizada para el desarrollo del proyecto es una metodología Scrum adaptada a las particularidades del proyecto. Una vez seleccionadas las herramientas de trabajo y llevada a cabo una primera versión del \emph{engine}, el desarrollo de cada modelo siguió un proceso iterativo. En términos de sprints serían los siguientes:
%% 
%% \lettrine{E}{l} comienzo de este proyecto consistió en una amplia búsqueda de aproximaciones y herramientas que orientasen la construcción del producto final.
%% 
%% También se hacía necesario decidir que tareas se considerarían como responsabilidad del software y cuales no. Se trataba de acotar el problema para obtener unos resultados satisfactorios. 
%% 
%% \section{Sprint a}
%% 
%% La primera de las tareas a abordar en el trabajo fue investigar la viabilidad del proyecto. Planteado el problema, no se conocía una solución directa. El resultado de esta primera fase de investigación fue conocer la existencia de los formatos para descripción del contenido del documentos digitalizados.
%% 
%% \begin{itemize}
%% 	\item Investigación de qué herramientas utilizar en Linux para tratar con ficheros PDF.
%% 	\item Información del formato HOCR
%% 	\item Información sobre el formato PDF
%% 	\item Uso de Tesseract
%% \end{itemize}
%% 
%% \section{Sprint b - Preparación del engine}
%% \begin{itemize}
%% 	\item Estructura de directorios
%% 	\item Creación del Makefile
%% 	\item Creación de los scripts
%% 	\begin{itemize}
%% 		\item Generación único para cada petición
%% 		\item Recepción de los ficheros
%% 		\item Renombrado de los fichero para tratamiento seguro
%% 		\item Discriminación de ficheros basados en texto y basados en imagen
%% 		\item Extracción de texto con pdftotext
%% 		\item Invocación de las aplicaciones para generar el lenguaje intermedio y el lenguaje estructurado
%% 	\end{itemize}
%% \end{itemize}
%% 
%% \section{Primer modelo}
%% \section{Segundo modelo}
%% \section{Tercer modelo}
%% \section{Modelo n...}
%% \section{Sprint f - Adaptaciones para la interfaz gráfica}
%% \begin{itemize}
%% 	\item \emph{Dockerizacion}
%% 	\item Generación de imágenes para cada página de los PDF basados en texto
%% 	\item Poblar el directorio de salida con los resultados
%% \end{itemize}
%% 
%% \section{Resultados de la planificación}
%% 
%% Aquí debe explicarse los desvíos en la planificación tanto temporales como económicos.

\section{Planificación temporal final}