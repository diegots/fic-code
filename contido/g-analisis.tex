%%%%           %%%%
%%%% ANÁLISIS  %%%%
%%%%           %%%%

\chapter{Análisis}
\label{chap:analisis}

\section{Actores}

\section{Requisitos funcionales}

\begin{itemize}
\item El sistema debe ser capaz de recepcionar 1 o múltiples ficheros por petición
\item El sistema debe ser capaz de tratar cada uno de los ficheros de forma independiente y generar su propia salida
\item El sistema debe ser capaz de tratar ficheros basados en texto o imagen
\item El sistema debe ser capaz de generar una salida en formato estructurado
\item El sistema debe ser capaz de generar imágenes de las páginas del documentos para disponibilidad del frontend
\item El sistema debe ser capaz de generar un identificador único para cada trabajo tratado
\item El sistema debe ser capaz de generar una marca indicativa del final de procesamiento
\item El sistema debe ser capaz de identificar líneas individuales en las tablas de los documentos. Aquí se puede hablar de por que
\end{itemize}

\section{Requisitos no funcionales}

\begin{itemize}
\item Mantener reducido el tiempo de cómputo    
\end{itemize}

\section{Casos de uso}

\section{Analisis de los documentos}

\subsection{Los distintos tipos de líneas}

\subsection{Tipos de regiones}

\subsection{Características de identificación}

Datos para identificar los documentos. En este caso, NIF, CIF.
