%%%%              %%%%
%%%% CONCLUSIONES %%%%
%%%%              %%%%

\chapter{Conclusiones}
\label{chap:conclusiones}

\lettrine{E}{ste} capítulo presenta las conclusiones finales del TFG y algunas líneas de mejora que se pueden abordar en el futuro.

Al comienzo de este proyecto se desconocían muchos aspectos del dominio. Gracias al trabajo realizado, ese conocimiento es ahora más completo y ha permitido crear un software capaz de resolver satisfactoriamente todos los objetivos inicialmente propuestos:

\begin{itemize}
	\item Tomando el conjunto de documentos de trabajo se han identificado sus regiones relevantes y el tipo de información que contienen.
	\item A partir del conocimiento de los documentos se han generado unas plantillas que caracterizan las partes relevantes.
	\item Se han resuelto las incógnitas iniciales para la obtención de las coordenadas que permiten realizar el emparejamiento entre el contenido y las plantillas.
	\item Se ha conseguido seleccionar la información relevante aplicando las plantillas.
	\item Se ha creado una herramienta de entorno servidor capaz de realizar un procesamiento por lotes automático.
	\item El software es capaz de tratar los documentos disponibles, de ambos orígenes, obteniéndose una salida estandarizada.
	\item Se ha colaborado para adaptar la aplicación de cara a su integración con una interfaz web.
\end{itemize}

\section{Lecciones aprendidas}

Como consecuencia de tratarse de un proyecto real, han existido muchas oportunidades para el aprendizaje. Además, la formación del Grado ha resultado de gran ayuda para completarlo. Estas son algunas de las lecciones aprendidas:

\begin{itemize}
	\item La historia del arte es muy útil para entender cuales son funcionalidades que aportan valor en otros trabajos y, por tanto, en cuales características interesa tomar en consideración.
	\item En varias ocasiones se planteó la cuestión de donde realizar cierta parte del tratamiento de un modelo, si en el generador de lenguaje intermedio o más adelante en Flex o Bison. Tomar la decisión no siempre es trivial y conlleva comprender mejor como se están repartiendo las responsabilidades en el software.
	\item La elaboración de la memoria implica una buena parte del trabajo pero avanzar en ella, a la par que en los demás aspectos del proyecto, ayuda a mejorar ambos.
	\item Combinar las responsabilidades de un trabajo y la realización del TFG no es siempre fácil. Respetando los horarios establecidos y siendo constante cada día, fue posible compaginar ambas.
\end{itemize}

\section{Trabajo futuro}

Hay funcionalidades que, estando fuera del alcance, han ido surgiendo y se podrían implementar en el futuro para enriquecer el proyecto:

\begin{itemize}
	\item Sería de gran utilidad disponer de un asistente para las plantillas. Consistiría en una vista donde se presenta el documento y se pueden seleccionar regiones de forma visual. Al mismo tiempo se podría asignar las caracterizaciones deseadas, como por ejemplo el tipo de región o su aplicación teniendo en cuenta el número de página del documento.
	\item Para reducir los tiempos de procesamiento sería interesante paralelizar las operaciones de extracción de texto y OCR. De forma análoga, la construcción del lenguaje intermedio y el procesamiento del lenguaje son tareas que se pueden realizar de manera simultánea para páginas y documentos diferentes.
	\item Se podría mejorar la gestión de errores para permitir mostrar información detallada de en caso de problemas con algún documento, así como exponer información para facilitar la generación de estadísticas.
\end{itemize}
