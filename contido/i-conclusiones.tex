%%%%              %%%%
%%%% CONCLUSIONES %%%%
%%%%              %%%%

\chapter{Conclusiones}
\label{chap:conclusiones}

\lettrine{E}{ste} capítulo presenta las conclusiones finales del proyecto y algunas líneas de mejora para el futuro.

Decidir en qué fase realizar un tratamiento determinado no es siempre trivial.

Los documentos deben ser tratados individualmente. Cada PDF no debe contener páginas mezcladas. Separar páginas de distintos documentos es otro problema en si mismo.

Para un correcto procesamiento los documentos que se proporcionen 
al sistema deben ser digitalizados.

Se espera que el tipo de regiones posibles en los documentos crezca pero se estabilice y los tipos existentes puedan ser reutilizados con mayor facilidad en nuevos tipos de documentos.

\section{Limitaciones de la solución}
La resolución seleccionada para las imágenes tiene gran implicación en varios aspectos.
\begin{itemize}
    \item Las plantillas solo representan una resolución determinada
    \item Imposibilidad de distinguir lineas distintas en las tablas
    \item Digitalización automática de los documentos para un mejor OCR
\end{itemize}

\section{Trabajo futuro}

\begin{itemize}
    \item Generación de coordenadas homogéneas para los casos procedentes de imagen y los procedentes de texto.
    \item Construcción de un asistente para generar las plantillas. Seleccionar el tipo de región y coordenadas de forma visual.
    \item Implementar un sistema para la notificación asíncrona a la finalización de los trabajos mediante un \emph{callback} que sea invocado a la finalización del trabajo.
    \item incorporación de una base de datos no relacionar para el almacenamiento de las plantillas.
    \item Paralelizar las operaciones de extracción de texto y OCR.
\end{itemize}

