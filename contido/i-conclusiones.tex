%%%%              %%%%
%%%% CONCLUSIONES %%%%
%%%%              %%%%

\chapter{Conclusiones}
\label{chap:conclusiones}

\lettrine{E}{ste} capítulo presenta las conclusiones finales del TFG y algunas líneas de mejora que se pueden acometer en el futuro.

Al comienzo de este proyecto se desconocían muchos aspectos del dominio. Gracias al trabajo realizado, ese conocimiento es ahora más completo y ha permitido crear un software capaz de resolver satisfactoriamente todos los objetivos propuestos en el Capítulo \ref{chap:introduccion}.

De forma general, se ha conseguido realizar la extracción del contenido vía OCR y/o texto. También se ha conseguido filtrar la información presente en los documentos gracias a unas plantillas que delimitan las regiones de interés. La salida creada está estandarizada, tanto en formato como en contenido.

Más concretamente, a partir del estudio de los documentos se ha conseguido identificar regiones equivalentes, aplicables a varios modelos e identificar el tipo de información que contienen.
Esta información se ha concretado en unas plantillas en formato JSON que el sistema toma como criterio para poder operar sobre los datos.
Gracias a la investigación sobre la salida de los sistemas de OCR y la constitución del formato PDF se ha conseguido comprender que la información de coordenadas está disponible y ha podido encontrar un método para explotarla en el trabajo. Se ha creado un software de entorno servidor capaz de procesar lotes de ficheros PDF y crear con ellos una salida ordenada y estructurada. La prueba de su funcionamiento se ha podido hacer con los documentos facilitados para el proyecto. Por último, se ha conseguido adaptar el desarrollo ya realizado para su incorporación con otra aplicación frontend de esta.

\section{Lecciones aprendidas}

El carácter abierto del proyecto, en lo que se refiere a su desarrollo, ha aportado muchas oportunidades para el aprendizaje. Se comentan algunas de ellas:

\begin{itemize}
	\item De forma tardía se descubrió importancia de realizar una historia del arte que permita comprender cuales son las funcionalidades y qué se esperaría de una solución en un entorno real. De otros trabajos existentes se puede aprender como enfocar correctamente un problema, ahorrando mucho trabajo.
	\item En varias ocasiones se planteó la cuestión de donde realizar cierta parte del tratamiento de un modelo, si en el generador de lenguaje intermedio o más adelante en Flex o Bison. Tomar la decisión no siempre es trivial y conlleva comprender mejor como se están repartiendo las responsabilidades en el software.
	\item No se debería desestimar la cantidad de trabajo que implica la elaboración de la memoria. Avanzar en ella a la par que en los demás aspectos del proyecto ayuda a mejorar ambos.
	\item Combinar el trabajo y la realización del TFG no ha sido fácil, pero respectando los horarios establecidos y siendo consistente cada día, fue posible compaginar ambas responsabilidades.
\end{itemize}

\section{Trabajo futuro}

Llegados a este punto quedan, muchas áreas de posible mejora o refinamiento, como por ejemplo:

\begin{itemize}
	\item Sería de gran utilidad disponer de un asistente para las plantillas. Consistiría en una vista donde se presenta el documento y se pueden seleccionar regiones de forma visual. Al mismo tiempo se podría asignar las caracterizaciones deseadas, como por ejemplo el tipo de región o su aplicación teniendo en cuenta el número de página del documento.
	\item Para reducir los tiempos de procesamiento sería interesante paralelizar las operaciones de extracción de texto y OCR. De forma análoga, la construcción del lenguaje intermedio y el procesamiento del lenguaje son tareas que se pueden realizar de manera simultánea para páginas y documentos diferentes.
	\item Se podría mejorar la gestión de errores para permitir mostrar información detallada de en caso de problemas con algún documento, así como exponer información para facilitar la generación de estadísticas.
\end{itemize}
