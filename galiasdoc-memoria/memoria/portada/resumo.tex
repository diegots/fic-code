%%%%%%%%%%%%%%%%%%%%%%%%%%%%%%%%%%%%%%%%%%%%%%%%%%%%%%%%%%%%%%%%%%%%%%%%%%%%%%%%

\begin{abstract}\thispagestyle{empty}

Este TFG tiene por objetivo la creación de una solución de entorno servidor, para automatizar la adquisición de información desde fuentes de datos semiestructuradas, como son los PDF. El proyecto se enmarca en el ámbito de la transición de organizaciones, públicas o privadas, hacia la informatización de los procesos.
Se centra específicamente en documentos donde existe un modelo común de maquetación, que los representan. Tanto si los ficheros PDF han sido creados digitalmente, por ejemplo, por un sistema de facturación, como si son simplemente producidos a partir de imágenes obtenidas por un escáner, desde documentos en papel, no resulta trivial recuperar de forma eficaz, la información contenida en ellos. El sistema realiza la extracción de la información, construye una representación interna de la misma gracias a sus coordenadas físicas, y posteriormente, la transforma a un formato estructurado, por medio de técnicas de procesamiento de lenguajes formales, en particular mediante los analizadores Flex y GNU Bison.

  \vspace*{25pt}
  \begin{segundoresumo}
This Bachelor's Degree Final Project aims to create a backend solution for automating information acquisition from semi-structured sources, like PDF file format. The project is framed within the scope of easing the transition of public or private organizations towards more process informatization.
It is specifically centered in the case of documents where there is a common layout that represents them. Whether the PDF files were digitally created, for example, by a billing system or they are simply sourced from images produced by a scanner, from paper documents, it is not trivial to effectively get back the information they hold. The system performs the information extraction, builds an internal representation thanks to the data's physical coordinates and finally, transforms that representation to a structured format, by employing formal languages processing techniques, particularly leaning on Flex and GNU Bison analyzers.
  \end{segundoresumo}

\newpage
\vspace*{25pt}
\begin{multicols}{2}
\begin{description}
\item [\palabraschaveprincipal:] \mbox{} \\[-20pt]
  \begin{itemize}
      \item Oficina sin papel 
      \item Lenguajes formales
      \item Flex
      \item Bison
      \item Transformada de Hough
      \item Procesamiento de Lenguajes
  \end{itemize}
%  \blindlist{itemize}[7] % substitúe este comando por un itemize
                         % que relacione as palabras chave
                         % que mellor identifiquen o teu TFG
                         % no idioma principal da memoria (tipicamente: galego)
\end{description}
\begin{description}
\item [\palabraschavesecundaria:] \mbox{} \\[-20pt]
  \begin{itemize}
      \item Paperless Office
      \item Formal languages
      \item Flex
      \item Bison
      \item Hough Transform
  \end{itemize}

  %\blindlist{itemize}[7] % substitúe este comando por un itemize
                         % que relacione as palabras chave
                         % que mellor identifiquen o teu TFG
                         % no idioma secundario da memoria (tipicamente: inglés)
\end{description}
\end{multicols}

\end{abstract}

%%%%%%%%%%%%%%%%%%%%%%%%%%%%%%%%%%%%%%%%%%%%%%%%%%%%%%%%%%%%%%%%%%%%%%%%%%%%%%%%
