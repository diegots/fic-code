\chapter{Fuentes de información propias}
\label{chap:preparación}

A la hora de ponerse a escribir la memoria, es útil disponer de registros o un histórico con todo aquello ocurrido e importante.

En este caso se dispuso de varias fuentes de información a las que recurrir:

\begin{itemize}
    \item Mensajes de correo electrónico intercambiados con el director y otras personas implicadas en el proyecto.
    \item Cuadernos manuscritos. Contienen notas elaboradas durante los momentos de trabajo y reuniones.
    \item Las conversaciones en formato chat mantenidas con Arturo Silvelo por medio de Teams, la herramienta que la UDC pone a disposición de los miembros de la Universidad.
\end{itemize}

Toda esta información de apoyo es útil al menos desde dos puntos de vista. Aquellos relacionados con los tiempos del proyecto: señalización de hechos importantes o hitos alcanzados. Y problemas resueltos. Esto es debido a que el registro de las dificultades acontecidas y el diseño de algoritmos particulares se hizo la mayor parte de las veces, en papel.

\section{Fechas relevantes}

Primera fase de desarrollo con los datos de Solco

\begin{center}
\begin{tabular}{ |c|c| } 
 04-10-2019 & Primera muestra de ficheros \\
 29-10-2019 & Primera reunión con los clientes \\
 03-12-2019 & Ángel crea el repositorio SVN \\
 03-02-2020 & Me doy cuenta de que no se espera se vaya a aplicar un modelo de aprendizaje \\
 06-02-2020 & 2º demostración en showroom del CITIC \\
\end{tabular}
\end{center}  

Comienzo del teletrabajo

\begin{center}
\begin{tabular}{ |c|c| }  
 08-01-2020 & Primera referencia a la estructura de directorios. \\
 13-01-2020 & Reunión de seguimiento con Victor, Dafonte. \\
 13-03-2020 & Cierra el CITIC por el Covid y nos envían para casa. \\
 02-04-2020 & Primer día en Odeene (jueves). \\
 14-04-2020 & Antonio Rojo facilita facturas Citic. \\
 14-04-2020 & Llegan las facturas de Betmedia. \\
 15-04-2020 & Planteamiento del nuevo modelo con los doc OVH. \\
 20-04-2020 & Firma del anteproyecto. \\
 21-04-2020 & Cambio al modelo de parsers como \emph{plugins}. \\
 28-04-2020 & Decisión de entrada separada por documento. \\
 27-04-2020 & Último commit en SVN. \\
\end{tabular}
\end{center} 

Segunda fase de desarrollo.

\begin{center}
\begin{tabular}{ |c|c| }  
 22-06-2020 & Primer commit del verano en git \\
 29-07-2020 & Primera reunión con Arturo \\
 12-08-2020 & Reunión con Arturo – confusión px vs. cm. \\
 17-08-2020 & Comienzo vacaciones Odeene \\
 19-08-2020 & Detectar que una linea ocupa varias columnas \\
 25-08-2020 & La aplicación funciona con Docker \\
 26-08-2020 & Petición cambios adaptación UI: directorio de salida único, imágenes de los PDF \\
 26-08-2020 & Cambio del jobId: fecha → timeStamp \\
 02-09-2020 & Las words llevan el nº de página \verb|p1w10| \\
 04-09-2020 & Fin vacaciones Odeene \\
 15-11-2020 & Último commit de esta fase
\end{tabular}
\end{center}

Más fechas por localizar
\begin{itemize}
    \item ¿Cuándo se me comunica que perdimos al primer cliente?
\end{itemize}

\begin{center}
    \begin{tabular}{c|c|c}
     25-11-2019 & 04-12-2019 & Primer repositorio Git \\
     03-12-2019 & 09-01-2020 & Segundo repositorio Git \\
     03-12-2019 & 27-04-2020 & Repositorio SVN \\
     22-06-2020 & 15-11-2020 & Último repositorio Git
    \end{tabular}
\end{center}

\section{Ideas importantes}

En la primera etapa de desarrollo había previsto crear unos \emph{tipos} de datos en la fase de generación del lenguaje intermedio. Estos tipos serían tratados luego por Flex y Bison. En la generación del lenguaje intermedio se identificaban las fechas, los números, y lo strings.

Aunque en el apartado de \ref{chap:implemetación} se explica el trabajo desde un punto de vista secuencial, lo cierto es que una vez seleccionadas las herramientas de trabajo y escrito en \emph{engine}, el resto de desarrollo es circular para cada uno de los nuevos modelos que hay que incorporar.

La \textbf{transformada de Hough} es un algoritmo que facilitó localizar lineas en dos de los modelos de documentos. Sin él no 

\underline{lunes, 13/01/2020}

Trabajos probablemente dirigidos por Dafonte:

\begin{itemize}
    \item TFG-INF 375
    \item TFG-INF 426
    \item TFG-INF 457
\end{itemize}

\underline{lunes, 20/01/2020}

Ejemplo de los tres tipos de lineas que hay en uno de los documentos:

\begin{itemize}
    \item Líneas de la factura
    \item Descripción de la agrupación por albarán
    \item Totales
\end{itemize}

Esta información, aunque fácil de distinguir para las personas, mucho más difícil de modelar.

\underline{lunes, 3/02/2020}

Inicialmente pensaba construir un único parser para todos los documentos. Al identificar cada documento sería fácil, tratarlo de forma individual. Nada más lejos de la realidad. Las colisiones en la gramática serían imposibles de resolver. Además con el enfoque final, se consigue un modelo comercial más adecuado: cada comprador de la aplicación recibe únicamente los parsers a los que tiene derecho.

Un fichero PDF es texto en formato ASCII de 7 bits. Todos los documentos contienen una cabecera que indica la versión del formato: \verb|%PDF1.7|.

\underline{lunes, 23/03/2020}

Mejora del tiempo utilizado para decidir si una word está dentro de una región de interés o no. La implementación inicial se basaba en el tipo Set. La nueva implementación tiene en cuenta las posiciones de la geometría de la palabra y la región.

\underline{(Probablemente) sabado, 11/07/2020:}

Flujo para el tratamiento de la información:

\begin{figure}[hp!]
  \centering
  \includegraphics[width=9cm]{imaxes/h-implementacion/flujo-información.png}
\end{figure}

Idea de un requisito: se desea mostrar los errores detectados, aunque no definen cuales errores. Pero cuidado, en esta herramienta no está incluido el frontend.

\underline{lunes, 13/07/20}

Idea de diseño: guardar los datos siempre en listas y procesar las listas. La ventaja es que si es una lista vacía no habrá salida, pero no hay que preocuparse sobre la existencia de los datos, como comprobaciones para evitar null pointers.

Un problema encontrado: una misma linea puede tener palabras a distintas alturas. Esto lleva a recordar que en el contexto de este desarrollo hay dos cosas distintas que se entienden por lineas. La primera es la idea natural de lo que es una linea para una persona que lee el documento. La segunda se refiere a la secuencia de palabras que conforman parte o la totalidad de la linea natural. Esto es así porque los contenidos de una linea natural pueden quedar separados en la extracción del texto. Puede haber otros contenidos intermedios. Esto ocurría por ejemplo con las facturas donde se listaban grupos de facturas. La siguiente situación donde puede ocurrir esto es debida a las columnas que trocean las lineas.

Sobre las regiones. Existen regiones cuyos contenidos podrían asimilarse a matrices. En particular 

\begin{math}
\begin{pmatrix}
  a_1 & a_2 & a_3 & a_4 \\
  b_1 & b_2 & b_3 & b_4 \\
  c_1 & c_2 & c_3 & c_4
\end{pmatrix}
\end{math}

La pregunta subyacente es, ¿puedo tratar del mismo modo los datos horizontales y verticales?

\underline{jueves, 23/07/2020}

Sobre los documentos multipágina:
\begin{itemize}
    \item Al mismo tiempo que se extrae el texto, se puede asociar el número de página.
    \item Luego se puede pasar el número como un parámetro.
    \item Si dos páginas son iguales, este conocimiento se define en Flex. Cuando se trata el tocket de la página $P_2$, sabiendo que es igual a la página $P_1$, se devuelve a Bison la $P_1$.
\end{itemize}

\underline{martes, 28/07/20}

Tecnologías para hacer un API intermedio entre la herramienta y el frontend. Pedro Lorenzo dio la idea de utilizar Node.js. Aunque finalmente se optó por no incluir ningún API entre el \emph{backend} y el \emph{frontend}.

\underline{miércoles, 29/07/2020}

Nombres de las empresas que facilitaron sus documentos para la elaboración de la herramienta:
\begin{itemize}
    \item Solco
    \item Betmedia
    \item Citic
\end{itemize}

\underline{sábado, 8/08/2020}

Se comparan las salidas XML y HOCR para una misma región simple y pequeña. No coinciden el orden de los elementos de salida.