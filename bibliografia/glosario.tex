%%%%%%%%%%%%%%%%%%%%%%%%%%%%%%%%%%%%%%%%%%%%%%%%%%%%%%%%%%%%%%%%%%%%%%%%%%%%%%%%
% Obxectivo: Lista de termos empregados no documento,                          %
%            xunto cos seus respectivos significados.                          %
%%%%%%%%%%%%%%%%%%%%%%%%%%%%%%%%%%%%%%%%%%%%%%%%%%%%%%%%%%%%%%%%%%%%%%%%%%%%%%%%

\newglossaryentry{bytecode}{
  name=bytecode,
  description={Código independente da máquina que xeran compiladores de determinadas linguaxes (Java, Erlang,\dots) e que é executado polo correspondente intérprete.}
}

\newglossaryentry{hipervisores}{
    name=hipervisor,
    description={Un hipervisor, \emph{hypervisor} en inglés, es un software utilizado para crear y ejecutar máquinas virtuales. Este tipo de software permite que un ordenador actuando de anfitrión, comparta recursos hardware con múltiples sistemas operativos invitados que están aislados los unos de los otros.}
}

